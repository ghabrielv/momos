%% Los cap'itulos inician con \chapter{T'itulo}, estos aparecen numerados y
%% se incluyen en el 'indice general.
%%
%% Recuerda que aqu'i ya puedes escribir acentos como: 'a, 'e, 'i, etc.
%% La letra n con tilde es: 'n.

\chapter*{Referencias Bibliográficas}

\noindent
% Aguilar, Ma.Guadalupe, José Carrillo, Antonio Rivera, and Victor González. 2006. “Growth Analysis and Sink-Source Relationships in Two Potato (Solanum tuberosum L.) Varieties.” Revista Fitotecnia Mexicana 29(2):145–56.\\

\noindent
% Available CRAN Packages By Name. Disponible en:\url{ https://cran.r-project.org/web/packages/available_packages_by_name.html}. Consultada Enero 2018.\\

\noindent
% Barraza, F.V., Fischer, G.,Cardona, C.E. (2004). Studying the process of tomato crop (Lycopersicon esculentum Mill.) growth in the
% Middle Sinu Valley, Colombia. Agronomía Colombiana 22 (1): 81-90 \\

\noindent
% Blackman, V.H. (1920). The Significance of the Efficiency Index of Plant Growth. New Phytologist, Vol. 19, No. 3/4, pp. 97-100\\

\noindent
% Borrego, Fernando et al. 2000. “Nota Técnica. Anáisis de Crecimiento En Siete Variedades de Papa (Solanum Tuberosum L.).” Agronomía Mesoaméricana 11(1):145-49.\\

\noindent
% Chambers, John M (1998). Programming with Data: A Guide to the S Language. Springer.\\

\noindent
% Condori, Bruno et al. 2008. “Agrophysiological Characterisation and Parametrisation of Andean Tubers: Potato (Solanum sp.), Oca (Oxalis tuberosa), Isaño (Tropaeolum tuberosum) and Papalisa (Ullucus tuberosus).” European Journal of Agronomy 28(4):526-40.\\

\noindent
% De Mendiburu F. (2017).  Package ‘agricolae’. cran.r-project.org. Disponible en:\url{ https://cran.r-project.org/web/packages/agricolae/agricolae.pdf}. Consultada Enero 2018.\\

\noindent
% Denny,M. (2017). The fallacy of the average: on the ubiquity, utility and continuing novelty of Jensen’s inequality. Journal of Experimental Biology 220, 139-146.\\

\noindent
% Dixon, P. (2003).VEGAN, a package of R functions for community ecology.Journal of Vegetation Science 14: 927-930.\\

\noindent
% Edmonson, R.N.(2014).Agridat.Journal of Agricultural Science , 152, 2.\\

\noindent
% Filippaa,G., Cremonesea,E., Migliavacca, m., Galvagnoa, M., Forkel,M., Wingate,L., Tomelleri,E., Morra,U., Richardsone,A.D. (2016). Phenopix: A R package for image-based vegetation phenology.Agricultural and Forest Meteorology 220, 141-150.\\

\noindent
% Gaitán, Ángela, María Gonzales, Carlos Ñústez, Tatiana Saldaña, and José Miguel Cotes. 2013. “Análisis Funcional de Crecimiento Y Desarrollo de Cuatro Variedades de Papa (Solanum tuberosum Subsp. andigena).” Revista Facultad de Ciencias Básicas 9(2):172-85.\\

\noindent
% Gardner, F., R. Pearce, and R. Mitchell. (1985). Physiology of Crop Plants. Iowa: Iowa State University Press.\\

\noindent
% Grime, J. and R. Hunt. (1975). “Relative Growth-Rate: Its Range and Adaptive Significance in a Local Flora.” Journal of Ecology 63(2):393-422.\\

\noindent
% Hendrik, Poorter. (1989). “Plant Growth Analysis: Towards a Synthesis of the Classical and the Functional Approach.” Physiologia Plantarum 75:237-44.\\

\noindent
% Horton, N., Kleinman, K. (2010). Using R and RStudio for Data Management, Statistical Analysis, and Graphics. CRC Press, A Chapman $\&$ Hall Book. USA. \\

\noindent
% Hunt, R. (1978). “Plant Growth Analysis: The Rationale behind the Use of Fitted Mathematical Function.” Annals of Botany 43:245-49.\\

\noindent
% Hunt, R. (1982). Plant Growth Curves: The Functional Approach to Plant Growth Analysis. New York: Cambridge University Press.\\

\noindent
% Hunt, R. (2002). “A Modern Tool for Classical Plant Growth Analysis.” Annals of Botany 90(4):485–88.\\ 

\noindent
% Hunt, R. (2003). “Growth Analysis, Individual Plants.” 579–88.\\

\noindent
% Kelley, C.T.(1987) Iterative Methods for Optimization, Society for Industrial and Applied Mathematics, Frontiers in Applied Mathematics.\\

\noindent
% Kniss, AR.,  Streibig JC. (2015). Statistical Analysis of Agricultural Experiments with R. (Disponible en:\url{ http://rstats4ag.org/}, Consultado: Enero 2018.\\

\noindent
% Larcher, W. (2003). Physiological Plant Ecology. Ecophysiology and Stress Physiology of the Functional Groups. Fourth. Berlin: Springer.\\

\noindent
%Leisch, F. (2002), “Sweave, Part I: Mixing R and LATEX,” R News, 2, 28–31, URL http://CRAN.R-project.org/doc/Rnews/.\\

\noindent
%Leisch F. (2009). Creating R Packages: A Tutorial.\\

\noindent
%Marsden J., Weinstein A. (1985). Calculus I. Springer-Verlag, New York Inc.\\

\noindent
% Park,B.B., Park,G.E.,Bae,K- (2015).Diagnosis of plant nutrient and growth responses on fertilization with vector analysis and morphological index.Forest Science and Technology, Vol. 11, No. 1,1-10.\\

\noindent	
% International Rice Research Institute (IRRI),  Biometrics and Breeding Informatics (BBI) group. Package STAT. Statistical Tool for Agricultural Research. Philippine  Disponible para descarga en:\url{ https://drive.google.com/file/d/0Bw-PTBz1SHmsM0VRTFc0MEZqQVU/view}. Consultado Enero 2018.\\

\noindent	
% Radford, PJ. 1967. “Growth Analysis Formulae - Their Use and Abuse.” Crop Science 7(3):171.\\

\noindent
% Salisbury, F.B. (1996). Units, Symbols, and Terminology for Plant Physiology.A Reference for Presentation of Research Results in the Plant Sciences.New York,Oxford, Oxford University Press.\\

\noindent
% Santos, Marcela. 2010. “Evaluación Del Crecimiento, Desarrollo Y Componentes de Rendimiento de Cuatro Cultivares de Papa Criolla En Dos Localidades Del Departamento de Cundinamarca.” Universidad Nacional de Colombia.\\

\noindent
% Sprouffske, K., Wagner,A. (2016). Growthcurver: an R package for obtaining interpretable metrics from microbial growth curves.Sprouffske and Wagner BMC Bioinformatics (2016) 17:172.\\

\noindent
% The R Foundation for Statistical Computing c/o Institute for Statistics and Mathematics Wirtschaftsuniversität Wien. (2014).  R: Software Development Life Cycle A Description of R’s Development, Testing, Release and Maintenance Processes. Disponible en:\url{ } . Consultado Enero 2018.\\

\noindent 
% Thornley, J., Johnson I.(1990). Plant and Crop Modelling: A Mathematical Approach to Plant and Crop Physiology. The Blackburn Press, New Jersey.\\

\noindent
%Trancón B.,  Carl  W., Bolz F.,  Grelck C. (2012).  The Functional Programming Language R and the Paradigm of Dynamic Scientific Programming. (182-197).\\

\noindent
%Wickham H. (2015). R Packages: Organize, Test, Document, and Share Your Code. 1 Edic. Hadley – Wickham.\\

\noindent
%Xie Y. (2017) . Package ‘formatR’. Disponible en:\url{ https://cran.r-project.org/web/packages/formatR/formatR.pdf}. Consultado Enero 2018.\\

\noindent
Martínez Rodríguez, Francisco y Otros: Lombricultura. Manual práctico, Impreso: Unidad de Producciones Gráficas MINREX, La Habana, Cuba, 2003.\\

\noindent
Rujano M., Ablan M., Sarmiento L. (2011). Simulación de la respuesta de la materia orgánica del suelo en diferentes ecosistemas ante escenarios de cambio climático en Venezuela. Disponible en: \url{http://erevistas.saber.ula.ve/index.php/ecodiseno/article/view/4372/4149}. Consultado Septiembre 2019.\\

\noindent
Universidad Complutense Madrid. Conservación de los recursos naturales para una Agricultura sostenible: Materia orgánica y actividad biológica. Disponible en: \url{https://www.ucm.es/data/cont/media/www/pag-104576/1.\%20Materia\%20org\%C3\%A1nica\%20y\%20actividad\%20biol\%C3\%B3gica.pdf}. Consultado Septiembre 2019.\\

\noindent
Tuomi, M., Vanhala, P., Karhu, K., Fritze, H., and Liski, J. (2008). Heterotrophic soil respiration- comparison of different models describing its temperature dependence. Ecol.Model., 211:182–190.\\

\noindent
Valery A.(20xx). La temperatura y humedad como reguladores de la descomposición de la MOS: desempeño de diversas funciones de respuesta en un gradiente altitudinal tropical.\\

\noindent
Ferrero R. (2018). Qué es R Software. Disponible en: \url{https://www.maximaformacion.es/blog-dat/que-es-r-software/}. Consultado Septiembre 2019.\\

\noindent
Darghan K. (2018). CGR Paquete en lenguaje R, para el cálculo de índices fisiológicos de crecimiento y componentes del rendimiento en plantas. (Trabajo Especial de Grado de pregrado). Universidad Nacional Experimental del Táchira. San Cristóbal, Estado Táchira.\\

\noindent
Contreras R.(2018). DiMBio, paquete en lenguaje R para la discriminación de modelos de simulación dinámicos para procesos biológicos. (Trabajo Especial de Grado de pregrado). Universidad Nacional Experimental del Táchira. San Cristóbal, Estado Táchira.\\
