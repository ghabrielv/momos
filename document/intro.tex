\chapter*{Introducci\'on}

\pagenumbering{arabic} % para empezar la numeración con números

El \textit{software} representa en la actualidad una herramienta para simplificar tareas que resultan complejas
y agotantes, por ejemplo el dar soluci\'on a un sinf\'in de c\'alculos estad\'isticos que un experto en una determinada \'area
emplear\'ia bastante tiempo en hallar. Situ\'andonos en el contexto de las ciencias agron\'omicas y \'areas afines, son diversos los procesos y c\'alculos
que van de la mano en el estudio de variables y modelos de toda \'indole que permitan dar respuesta a una situaci\'on dada.\\

El objetivo del presente estudio es simular el modelo MOMOS (Micro - Organismos y Materia Orgánica del Suelo) creado por Pansu \textit{et al.}, (2004), con el cual se simula la din\'amica del carbono del suelo, centr\'andose en el funcionamiento de los microorganismos que descomponen como los principales responsables de los procesos metabólicos que producen los cambios de esta materia orgánica.\\

El modelo MOMOS fue modificado por Valery (2018) y reproducido en VENSIM que es una herramienta visual de modelaje que permite conceptualizar, documentar, simular, analizar y optimizar modelos de dinámica de sistemas, sin embargo, esta herramienta es privativa y limitada. Con la presente investigaci\'on se busc\'o crear un paquete en lenguaje R que permita la simulaci\'on del modelo din\'amico MOMOS, dando como ventaja que es un \textit{Software Libre} y tiene una gran cantidad de colaboradores en el \'area de las ciencias m\'atematicas y de computaci\'on.\\
