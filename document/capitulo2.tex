\chapter{Fundamentos Te\'oricos}

\section{Antecedentes}

Contreras (2018)  construy\'o: “DimBio, Paquete en lenguaje R para la discrminizaci\'on de modelos de simulaci\'on din\'amicos para procesos biol\'ogicos.”, en este trabajo se expone de forma detallada la realizaci\'on de un paquete en lenguaje R, para la discriminaci\'on de modelos de simulaci\'on din\'amicos para procesos biol\'ogicos, para unificar los estad\'isticos que evaluan el desempeño de los modelos din\'amicos, mediante el c\'alculo para discrimaci\'on de estad\'isticos univariantes y multivariantes utilizados para encontrar un modelo adecuado o que mejor se ajustar\'a al proceso de estudio basado en modelos.\\ 

La investigaci\'on fue de relavancia para este trabajo de grado, ya que en ella crearon un paquete en R, y utilizaron modelos de simulaci\'on din\'amicos.\\

Darghan (2018) realiz\'o: “CGR, Paquete en lenguaje R, para el cálculo de índices fisiológicos de crecimiento y componentes del rendimiento en plantas.” trabajo en el cual se expone el desarrollo del paquete CGR cuyo interés radicó en el cálculo de aquellos índices que dependen de la primera derivada de la forma funcional ajustada al modelo de crecimiento de las plantas, también permite el cálculo de índices instantáneos en cualquier punto de interés por el usuario dentro de la región experimental o dentro del dominio de la función ajustada.\\

La investigaci\'on citada fue tomada como referencia, puesto que realizó cálculos de índices para evaluar el comportamiento de ciertas variables en procesos biol\'ogicos, en este caso, el crecimiento. Lo cual est\'a vinculado al contexto biol\'ogico de los datos que fueron evaluados por el paquete que se desarroll\'o.\\

Ablan (2011) construy\'o: “Una librer\'ia en R para validaci\'on de modelos de simulaci\'on”, en este trabajo se manejaron m\'etodos para la validaci\'on de modelos de simulaci\'on continua para evaluar la calidad del modelo comparando los resultados de series temporales de datos reales con los resultados o salidas obtenidos de una simulaci\'on para el mismo lapso temporal, en el cual se utilizaron los criterios o \'indices de validaci\'on que pueden ser usados para la comparaci\'on de datos y resultados de un modelo, los cuales fueron: el error cuadr\'atico medio, el estad\'isticos de Theil, el error absoluto medio, el \'indice de acuerdo de Willmott, la eficiencia del modelo de Nash Sutcliffe, el coeficiente de correlaci\'on de Pearson, el coeficiente de determinaci\'on, la prueba F, la prueba t pareada, el estad\'istico Hotelling T2, el criterio de informaci\'on de Akaike y la inferencia bayesiana. Con estos criterios o \'indices se cre\'o una librer\'ia del programa estad\'istico R que permiti\'o el uso de estos \'indices para la validaci\'on de modelos de simulaci\'on.\\

La investigaci\'on fue importante en este trabajo de grado, ya que en ella crearon una librer\'ia de R, y utilizaron m\'etodos estad\'isticos para respuesta univariante que son utilizados para la comparaci\'on de datos reales contra datos de modelos de simulaci\'on.\\

\section{Bases Te\'oricas}

\section{Modelos}

En la actualidad el avance de la tecnolog\'ia, permite que para el estudio de sistemas biol\'ogicos,  los modelos de simulaci\'on sean una herramienta que ayuda en el manejo de la informaci\'on, as\'i como la evaluaci\'on y planificaci\'on de manejos y dise\~no de experimentos.\\

Existe una gran cantidad de modelos biol\'ogicos, los cuales est\'an relacionados con diversos procesos en los ecosistemas o agroecosistemas. Entre estos se pueden observar modelos te\'oricos, diagram\'aticos o matem\'aticos. “Estos modelos han venido evolucionando desde la \'ultima d\'acada, buscando entender los procesos que permiten el almacenamiento o la p\'erdida del suelo.” (Manzoni, 2009).\\

El uso de modelos en el \'area ecol\'ogica puede estar enfocado en diversos objetivos, entre los que se pueden destacar: la representaci\'on de diversas variables de estudio y sus tasas de cambio; la descripci\'on de ecosistemas y los procesos temporales y/o espaciales de los mismos; evaluar las condiciones pasadas y predecir el comportamiento futuro de los ecosistemas en estudio; poner a prueba teor\'ias o hip\'otesis sobre la estructura y el funcionamiento de los ecosistemas (Blanco, 2013).\\

Dentro del grupo de modelos que estudian las transformaciones de la materia org\'anica del suelo (MOS), se presentan caracter\'isticas similares entre ellos, tales como, estar conformados por compartimientos, que representan a los diferentes compuestos qu\'imicos o biol\'ogicos, tambien  se pueden diferenciar, por el n\'umero de compartimientos que poseen, que pueden ir desde 2 hasta 200 compartimientos. Sin embargo, ''debe tenerse en cuenta un equilibrio entre el detalle y la comprensi\'on de la informaci\'on, para no perder el sentido de la modelizaci\'on y poder realizar simulaciones y predicciones acordes a la realidad'' (Blagodatsky 1998; Bruun 2003; Kirschbaum 2002;  Zhang 2008).\\


La literatura presenta el inter\'es que existe por los modelos que tienen que ver con las transformaciones de la MOS, muestra de esto es la gran cantidad de modelos que son referenciados en el registro de modelos ecol\'ogicos (REM \url{http ://ecobas.org}), donde se presenta la informaci\'on de cerca de 684 modelos.\\

Seg\'un lo presentado por Haefner (2005), sobre la importancia de los modelos y su potencial para el estudio de sistemas naturales, se tiene que mediante los mismos se puede lograr:\\


\begin{enumerate}
  \item Sintetizar, manejar y unificar una gran cantidad de informaci\'on en forma de ecuaciones y sus relaciones, permitiendo entender y explicar el comportamiento de los sistema biol\'ogicos.
 \item Analizar el funcionamiento del sistema para predecir el comportamiento futuro, bajo las condiciones naturales o con intervenci\'on de manejo.
\end{enumerate}

``Los modelos de simulaci\'on plantean un esquema b\'asico de trabajo, el cual es fundamentado en tres etapas generales'' (Haefner, 2005; Liu 2005), las cuales pueden ser descritas de la siguiente manera:\\

1.- Conceptualizaci\'on. Conocer en gran medida la realidad que se trata de modelizar, ser capaces de representar l\'ogicamente y de manera conceptual el problema. \\

2.- Formalizaci\'on. establecer correctamente las relaciones entre los elementos que conforman el sistema en estudio, de forma que sean comprensibles, lo que permite construir un diagrama de estas relaciones y posteriormente representarlo mediante un diagrama de \textit{Forrester}.\\

3.- Evaluaci\'on. Establecer la forma en que debe ser el procedimiento de resoluci\'on a emplear, y la manera de interpretarlo correctamente.\\


Las etapas de conceptualizaci\'on y formalizaci\'on se basan en toda la informaci\'on de d\'ecadas pasadas, sin embargo, el proceso de evaluaci\'on presenta diversos problemas, entre los que destacan la recolecci\'on de informaci\'on con caracter\'isticas confiables que sea representativa de los sistemas y validar los resultados obtenidos por el modelo con los obtenidos experimentalmente mediante un an\'alisis estad\'istico inferencial.


\subsection{Lenguaje R.}

R es un conjunto  integrado de \textit{software} de c\'odigo abierto para el almacenamiento, manipulaci\'on, c\'alculo y visualizaci\'on de datos para computaci\'on y graficaci\'on estad\'istica, puede ser compilado y ejecutado en  en Windows, Mac OS X y otras  plataformas UNIX (como Linux), se distribuye usualmente en formato binario (\url{https://www.r-project.org/about.html}, 2018). El proyecto de \emph{software} R fue iniciado por Robert Gentleman y Ross Ihaka. El lenguaje fue influenciado por  lenguaje S desarrollado originalmente en Bell Laboratories por John Chambers y sus colegas. Desde entonces ha evolucionado  para el c\'alculo estad\'istico asociado a diversas disciplinas para contextos acad\'emicos y comerciales. En R, la unidad fundamental de c\'odigo compartible es el paquete, el cual agrupa c\'odigo, datos, documentaci\'on y pruebas, y resulta simple de compartir con otros. Para enero del 2015 ya hab\'ian m\'as de 6.000 paquetes disponibles en la Red Integral de Archivos de R, conocido com\'unmente por su acr\'onimo CRAN, el cual es el repositorio de paquetes. Esta gran variedad de paquetes es una de las razones por las cuales R es tan \'exitoso, pues es probable que alg\'un investigador o acad\'emico, ya haya resuelto un problema en su propio campo usando esta herramienta, por lo que otros usuarios simplemente podr\'an recurrir a ella para su uso directo o para llamarla en un nuevo c\'odigo (Wickham,2015). \\


\subsection{RStudio.}

RStudio es un ambiente de desarrollo integrado (\textit{Integrated Development Environment}, IDE) que ofrece herramientas de desarrollo vía consola, editor de sintaxis que apoya la ejecución de código, así como herramientas para el trazado, la depuración y la gestión del espacio de trabajo.  RStudio está disponible para Windows, Mac y Linux o para navegadores conectados a RStudio Server o RStudio Server Pro (Debian / Ubuntu, RedHat / CentOS, y SUSE Linux) (\url{https://www.rstudio.com/about/}, 2018).
 

\subsection{Estructura de paquetes en R/RStudio.}

La estructura de un paquete en R cuenta con al menos los cuatro \'items siguientes:

\begin{enumerate}
  \item \textbf{ DESCRIPTION:} En este componente se encuentra la metadata del paquete. La tarea del archivo Description es de gran importancia, ya que es en donde se registra la metadata, las dependencias que utiliza el paquete, la licencia y el soporte en caso de ocurrir errores con el mismo.\\

La estructura m\'inima para realizar un DESCRIPTION de un paquete en R es la siguiente:

\begin{itemize}
\item Package: mypackage
\item Title: What The Package Does (one line, title case required)
\item Version: 0.1
\item Authors@R: person("First", "Last", email = "first.last@example.com",
\item role = c(``aut", ``cre"))
\item Description: What the package does (one paragraph)
\item Depends: R (>= 3.1.0)
\item License: What license is it under?
\item LazyData: true
\end{itemize}

 \item\textbf{ R/Directorio:}  Direcci\'on del repositorio donde se encuentra el c\'odigo del paquete, es decir los  ``.R'' archivos. Se exponen las buenas pr\'acticas a la hora de realizar todo el c\'odigo en R, desde organizaci\'on de las funciones, estilos de c\'odigo, comentarios y nombre de variables. 
\begin{itemize}
\item Organizar funciones en R: Aunque se puede organizar los archivos como se desee, los dos extremos son malos, no colocar todas las funciones en el mismo archivo y no crear un archivo para cada funci\'on, aunque si una funci\'on es muy grande o tiene mucha documentaci\'on se puede dar el caso. Los nombres de los archivos tienen que ser significado y deben de terminar en R. Se puede recomendar de acuerdo al n\'umero de funciones utilizar prefijo.

\item Nombres de objetos: Los nombres de las variables y funciones deben de ser en min\'usculas, usar el gui\'on bajo ( \_ ) para separar palabras. En lo posible no debe usarse nombres de variables existentes, esto causar\'a confusi\'on.

\item Comentarios: Para comentar el codig\'o, se comienza con \#, los comentarios deben de explicar el porqu\'e, no el que. Se usan los caracteres (-) y (=) para separar l\'ineas.\\

\item Estilos de c\'odigo: Existe diferencia entre el c\'odigo utilizado en \textit{scripts} y paquetes:\\
en un \textit{script}, el c\'odigo se ejecuta cuando se carga, mientras en un paquete el c\'odigo se ejecuta cuando se genera. Esto significa que el c\'odigo de paquetes solo debe crear objetos, a continuaci\'on se ampl\'iam estas importantes diferencias:

\begin{itemize}
\item Cargando c\'odigo: Cuando se carga un \textit{script} el c\'odigo se ejecuta de una vez, el procedimiento es diferente en un paquete, porque es cargado en dos pasos, el primero, cuando se construye el paquete todo el c\'odigo se ejecuta en R/ y su resultado es guardado, el otro paso es cuando se carga un paquete, con \textit{library()} o \textit{require()}, los resultados almacenados en la cach\'e est\'an dispuestos para su uso.
\item The R landscape: Hay tambi\'en diferencia en un \textit{script} y un paquete, por lo tanto hay que prestar atenci\'on a R \textit{landscape}, incluyendo las funciones, objetos y todas las variables globales.
\end{itemize}
\end{itemize}
Recomendaciones:

\begin{itemize}
\item  No se suele utilizar library() o require(). Estos modifican la ruta de b\'usqueda, lo que afecta las funciones disponibles del entorno global. Es mejor usar la descripci\'on para especificar los requisitos de un paquete.
\item Nunca emplear \textit{source()} para cargar el c\'odigo de un archivo. Ya que \textit{source()} modifica el entorno actual, insertando los resultados de ejecutar el c\'odigo. En cambio, usando el devtools::load\_all()  autom\'aticamente se generar\'an todos los archivos en R.
 \end{itemize}

 \item \textbf{Man/Documentaci\'on:} La documentaci\'on es uno de los aspectos m\'as importantes de un buen paquete. Sin ella, los usuarios no sabr\'an c\'omo usar un paquete y adem\'as  ayudan a recordar que realizan sus funciones para el futuro.

R proporciona una forma est\'andar de documentar los objetos en un paquete: Se escriben  archivos .Rd en el directorio ``man''. \'Estos tilizan LaTeX, y se procesan en HTML, texto plano y pdf para su visualizaci\'on. En lugar de escribir estos archivos a mano, se emplea ``roxygen2'', que convierte los comentarios especialmente formateados en archivos ``.Rd'',  el objetivo de roxygen2 es hacer que documentar el c\'odigo sea lo m\'as f\'acil posible.

Los comentarios roxgen utilizan \# si se utilizan para distinguir de los comentarios regulares.

Para documentar funciones se utilizan generalmete 3 etiquetas que son:
\begin{itemize}
	\item @param: donde se describe los par\'ametros de entrada de la funci\'on.
	\item @examples: donde se muestra un ejemplo funcional de la funci\'on ya que R CMD chequea que sea correcto.
	\item @return: espec\'ifica lo que retorna la funci\'on siempre que sea necesario.
 \end{itemize}

Para documentar el paquete se recomienda utilizar roxygen con la etiquetas @docType package y @name (nombre del paquete) tambi\'en se puede utiliza la etiqueta @section que permite ser espec\'ificos al momento de documentar.

Documentaci\'on de clases, gen\'ericos y m\'etodos:
\begin{itemize}
	\item S3: La clase S3 no tiene definici\'on formal as\'i que documenta la funci\'on constructora. No se necesita documentar m\'etodos simples pero si el m\'etodo es complicado o incluye argumentos adicionales se debe de documentar.
	\item S4: La clase S4 utiliza roxygen para realizar la documentaci\'on, los S4 tambi\'en son funciones as\'i que se deben documentar como tal, la diferencia con S3 es que todos los m\'etodos deben documentarse. A menudo el c\'odigo S4 necesita ejecutarse en cierto orden, por ejemplo, para definir los m\'etodos ``setMethod(``foo", c(``bar",``baz"), ...)''  se deben haber creado antes el ``foo'' gen\'erico y las 2 clases.
	\item RC: Las clases RC de referencia son diferentes a S3 y S4 porque los m\'etodos est\'an asociados con clases, no con gen\'ericos. La ``docstring'' es una cadena colocada dentro de la definici\'on del m\'etodo que describe brevemente lo que hace. Esto hace que documentar RC sea m\'as simple que S4 porque solo se necesita un bloque de roxygen por clase.
 \end{itemize}

  \item\textbf{ NAMESPACE:} Espec\'ifica que objetos conforman el paquete. Se utiliza para  proporcionar espacios a los nombres como su nombre lo indica, es utilizado para interactuar con los paquetes y sus variables del CRAN de R.\\
\end{enumerate}


\subsection{Glosario}

\paragraph{Crecimiento}: Aumento en un atributo medido, $x$, de un \'organo u organismo, como una funci\'on del tiempo, $t;x = f (t)$.
\paragraph{Lenguaje R}: R es un entorno y lenguaje de programaci\'on con un enfoque al an\'alisis estad\'istico.
\paragraph{R Studio }: R Studio es un entorno de desarrollo integrado, en ingl\'es ``integrated development environment" (IDE) para R.


\section{\textit{GNU General Public License v2.0 (GPL-2.0) }}

	Es una licencia de \textit{software} libre, que garantiza se pueda copiar, distribuir y modificar el \textit{software} siempre que realice un seguimiento de los cambios / fechas en los archivos de origen. Cualquier modificación o software que incluya (a trav\'es del compilador) el código con licencia GPL también debe estar disponible bajo la GPL junto con las instrucciones de compilación e instalación.