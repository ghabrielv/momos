\chapter{Fundamentos Metodol\'ogicos}

	A continuaci\'on se plantea la metodolog\'ia para el presente trabajo, detallando el enfoque, tipo, nivel y dise\~no de la investigaci\'on y la metodolog\'ia que se implementa.
	
\section{Enfoque de la investigaci\'on}
	
	La presente investigaci\'on se desarrolla siguiendo un enfoque cuantitativo, puesto que, como lo indican Pallela y  Martins (2012) , “la investigaci\'on cuantitativa requiere el uso de instrumentos de medici\'on y comparaci\'on, que proporcionan datos cuyo estudio necesita la aplicaci\'on de m\'odelos matem\'aticos y estad\'isticos, el conocimiento est\'a basado en hechos”.  Los datos usados en el desarrollo del paquete y para la comparacion de los resultados, provienen de la tesis Doctoral de Valery (2018).\\
	
\section{Tipo o nivel de investigaci\'on}
	
	Este proyecto plante\'o un tipo de investigaci\'on de desarrollo, en donde se integra la programaci\'on y evaluaci\'on del comportamiento de un paquete de simulaci\'on, que permite indagar los efectos de la interrelaci\'on entre los diferentes tipos de variable en lugar de los hechos.\\

	En este punto se determin\'o la profundidad que abarca esta investigaci\'on, teniendo en cuenta que de acuerdo con  el nivel de la investigaci\'on es definido como “grado de profundidad con que se aborda un fen\'omeno u objeto de estudio” (Arias, 2012).\\

	En este sentido, se tiene que dadas las caracter\'isticas del proyecto, se asocia con un nivel descriptivo, tal como lo indican Pallela y  Martins (2012),  “hace \'enfasis sobre conclusiones dominantes o sobre como una persona, grupo o cosa se conduce o funciona en el presente” esto debido a que se midieron los datos extra\'idos sin alterarlos para ser mostrados en el sistema.\\

\section{Dise\~no de la investigaci\'on}
	
Seg\'un Arias(2012), el dise\~no de la investigaci\'on es “la estrategia general que adopta el investigador para responder al problema planteado” (p.21) por lo que es vital haber establecido una correcta secuencia de pasos para la elaboraci\'on del prototipo de software que dio soluci\'on a la problem\'atica principal de la investigaci\'on.\\

Con este enfoque, se tiene que este trabajo sigui\'o un dise\~no no experimental, enfocado en el uso de informaci\'on existente, de acuerdo con lo dicho por Pallela y  Martins (2012) al definir el dise\~no no experimental como:\\

Es el que se realiza sin manipular en forma deliberada ninguna variable. El investigador no sustituye intencionalmente las variables independientes. Se observan los hechos tal y como se presentan en su contexto real y en un tiempo determinado o no, para luego analizarlos. Por lo tanto, este dise\~no no se construye una situaci\'on espec\'ifica sino que se observan las que existen. Las variables independientes ya han ocurrido y no pueden ser manipuladas, lo que impide influir sobre ellas para modificarlas. (p.81)"\\

Esto indica que no hay manipulaci\'on de variables. Esta investigaci\'on presenta una modalidad de proyecto especial que, como lo indican Pallela y  Martins (2012), los proyectos especiales “destinados a la creaci\'on de productos que puedan solucionar deficiencias evidenciadas, se caracterizan por su valor innovador y aporte significativo” (p.92), ya que se cre\'o un \emph{software} aplicable al \'area de estudio.\\

\section{Metodolog\'ia}

Para el desarrollo del paquete  se sigui\'o las pautas est\'andar establecidas para la creaci\'on de paquetes y extensiones en R.\\

\noindent
\textbf{Creaci\'on del esqueleto del paquete.}\\


En esta etapa se dise\~naron y crearon los directorios, ficheros y objetos que conforman el paquete.\\ 

\noindent
\textbf{Registrar el m\'etodo para el env\'io y uso de funciones.}\\

En esta etapa del desarrollo se estableci\'o las dependencias sobre los paquetes de la base fuente de c\'odigo R y sus m\'etodos de conexi\'on, considerando el manejo de versiones y los criterios de mantenimiento, adem\'as  se estableci\'o  los espacios de nombre o las estrategias para la b\'usqueda y utilizaci\'on de las variables; unificando estos criterios a las funciones que fueron dise\~nadas.\\

\noindent
\textbf{Dise\~no y codificaci\'on de las funciones.}\\

Los m\'etodos para el dise\~no de las funciones primarias en R fueron los diagramas de flujo; y para su codificaci\'on se siguieron las normas de estilo para codificaci\'on en R, sugeridas por Wickham (2015) y por el creador del paquete \emph{formatR} Xie(2017), adem\'as se estableci\'o la dependencia con las funciones de c\'odigo base y las recomendadas para desarrollo en R.\\

\noindent
\textbf{Pruebas unitarias de las funciones.}\\

Debido a que los paquetes en R est\'an conformados, entre otros elementos por las funciones primarias, a cada una de ellas se les realizaron pruebas unitarias en dos fases, la primera con datos sint\'eticos que permiten comprobar cada estado del diagrama de flujo, que esquematiza la soluci\'on num\'erica que permite el c\'alculo, para esto se utiliz\'o el paquete de R\emph{RUnit} (Zenka, 2015) y \emph{testthat} (Wickham, 2017), y la segunda etapa donde el modelo final ya implementado permite realizar pruebas con los conjuntos de datos facilitados que formaron parte integral del paquete y con los cuales se desarrollaron los ejemplos pr\'acticos que conformaron la documentaci\'on que acompa\~a al paquete R.\\


\newpage  
\noindent
\textbf{Chequear la carga del paquete.}\\

En esta etapa del desarrollo se utiliz\'o las funciones de chequear el paquete que ofrece el c\'odigo R; cuya finalidad es verificar cada fichero del \'arbol de carpetas asociadas a cada elemento de la estructura o esqueleto del paquete, que a su vez crea el archivo de documentaci\'on en LaTeX y/o HTML, compila el c\'odigo fuente y crea las librerías de enlace din\'amico (\emph{dynamic link library} DLL).\\  

\noindent
\textbf{Construcci\'on del m\'etodo de distribuci\'on del paquete.}\\

Se seleccion\'o la forma de distribuci\'on del paquete desde el repositorio local, creando los ficheros fuentes  (en formato  \emph{tarball})  y en binario.\\

\section{Aspectos administrativos}

\vspace{1 cm}
La realizaci\'on de la investigaci\'on fue planificada seg\'un lo establecido en el siguiente diagrama:\\

\begin{figure}[!ht]
\begin{center}

\begin{ganttchart}[y unit title=0.4cm,y unit chart=0.5cm,vgrid,hgrid,title height=1,bar/.style={draw,fill=cyan},bar incomplete/.append style={fill=yellow!50},bar height=0.7]{1}{16}
 \gantttitle{Pasos}{16}\\
 \ganttbar{\tiny{Determinaci\'on de los lineamientos para el dise\~no del paquete}}{1}{2} \\
 \ganttbar{\tiny{Creaci\'on del esqueleto del paquete}}{3}{3} \\
 \ganttbar{\tiny{Dise\~no de los algoritmos para la biblioteca de funciones}}{3}{6} \\
 \ganttbar{\tiny{Codificaci\'on de la biblioteca de funciones}}{6}{10} \\
 \ganttbar{\tiny{Prueba de la biblioteca de funciones}}{9}{11} \\
 \ganttbar{\tiny{Documentaci\'on del paquete}}{4}{12} \\
 \ganttbar{\tiny{Chequeo del paquete}}{10}{12} \\
 \ganttbar{\tiny{Creaci\'on del m\'etodo de distribuci\'on del paquete}}{13}{14} \\
 \ganttbar{\tiny{Pruebas de distribuci\'on del paquete}}{15}{16} \\
%  \ganttbar[progress=70]{Fase 3}{13}{18} \\
 % \ganttbar[progress=40]{Conclus\~ao}{20}{24} \\
 \ganttbar{\tiny{Realizaci\'on del informe del proyecto especial de grado}}{2}{16} \\
 \ganttlink{elem0}{elem1}
 \ganttlink{elem6}{elem7}
 \ganttlink{elem7}{elem8}
\end{ganttchart}

\end{center}
\caption{Diagrama de Gantt con la planificaci\'on del proyecto especial de grado}
\end{figure}
