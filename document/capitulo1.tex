\chapter{El Problema}

\section{Planteamiento y formulaci\'on del problema}
La materia org\'anica estable del suelo (MOS) se caracteriza por ser el resultado del proceso de descomposici\'on de los residuos de animales y vegetales, los cuales varían en proporci\'on y estado. Estos residuos son sustancias que contribuyen a la fertilidad del suelo, por lo que,  la materia org\'anica del suelo MOS, viene a ser uno de los factores de importancia que determina localidad de un suelo.\\

La descomposici\'on de la materia org\'anica es un proceso biol\'ogico que ocurre naturalmente, por efecto de diferentes grupos de organismos, entre los cuales destacan los microorganismos del suelo, que permiten transformar los diferentes residuos, en nutrientes b\'asicos y energía. Siendo que una de las funciones MOS es actuar como dep\'osito de carbono, permitiendo que exista en el suelo m\'as del doble del carbono contenido en la atm\'osfera, por lo que los cambios que experimente est\'a, pueden tener un impacto en el equilibrio global, y por lo tanto un efecto sobre las condiciones clim\'aticas del planeta, al ser considerado el carbono que sale de esta como CO2, el cual es uno de los gases que contribuyen al efecto invernadero del planeta y al cambio clim\'atico (IPCC, 2000).\\

En la actualidad, se reporta un incremento en el cambio clim\'atico por efecto del impacto antropog\'enico, el cual se observa en el aumento de la temperatura en diferentes regiones, y como efecto de retroalimentaci\'on la temperatura influye sobre el proceso de descomposici\'on y por lo tanto en el almacenamiento de carbono en el suelo y en la vegetaci\'on (Andriulo, 2006), todo esto muestra que el suelo es un compartimiento complejo debido a todas las interrelaciones que existen en el mismo.\\

Por lo que la comprensi\'on de los diferentes procesos y transformaciones que ocurren en el suelo, son de gran importancia, sin embargo la cantidad de informaci\'on que puede ser recolectada, así como la variabilidad en la misma, constituyen un desafío en sí, por lo cual esto conlleva ha utilizar herramientas tecnol\'ogicas que permitan el manejo de la informaci\'on obtenida, es por ello que existe una gran cantidad de modelos que simulan y describen los procesos de la descomposici\'on de la MOS, y los cuales pueden o est\'an siendo acoplados a modelos de pron\'osticos a nivel global, de igual manera estos modelos est\'an evolucionando, en la b\'usqueda de una mejor comprensi\'on de los diversos procesos físico, químicos y biol\'ogicos [Larionova et al., 2007; Lomander et al., 1998].\\

Existen una gran cantidad de modelos que estudian la MOS, muchos de los cuales est\'an desarrollados sobre plataformas propias, lo cual nos permite el proceso de enseñanza-aprendizaje, así como, de investigaci\'on en el desarrollo de mejoras al mismo por otros investigadores, de igual manera existen herramientas donde se han desarrollado de manera pr\'actica modelos, tales como por ejemplo el modelo MOMOS (Materia Organica y MicroOrganismos del Suelo), el cual ha sido calibrado y validado con datos e informaci\'on de un estudio en un gradiente altitudinal tropical [Pansu et al., 2010, 2014] mostrando buenos resultados en las diferentes evaluaciones, este mismo modelo ha sido evaluado, encontrando una propuesta de evoluci\'on del mismos (Valery, 2018). Sin embargo, las investigaciones antes mencionadas, presenta la desventaja de haber sido desarrolladas en un software de simulaci\'on privativo llamado Vensim creado por Ventana Systems, Inc. para la construcci\'on de modelos de sistemas din\'amicos, limitado a funciones desarrolladas por dicha empresa, lo que no permite ampliar la investigaci\'on en otros aspectos de inter\'es sin la necesidad de acudir a otros softwares externos.\\

De igual manera, el grupo de Investigaci\'on en Biotecnología Agrícola y Ambiental (GIBAA) de la UNET, ha desarrollado diversos planteamientos de modelos bajo un software libre, por lo que se hace uso de la herramienta para el an\'alisis estadístico y gr\'afico llamado R, el cual es un ambiente de programaci\'on formado por un conjunto de herramientas muy flexibles que pueden ampliarse f\'acilmente mediante paquetes, librerías o definiendo funciones propias. Adem\'as, es gratuito y de c\'odigo abierto. Cualquier usuario puede descargar y crear su c\'odigo de manera personal, sin restricciones de uso, la \'unica regla es que la distribuci\'on siempre sea libre (GPL). Gracias a que puede accederse libremente a su c\'odigo, R no tiene limitadas sus funciones, al contrario de lo que sucede con otras herramientas comerciales. \\

La finalidad de esta investigaci\'on consiste en utilizar el modelo MOMOS desarrollado por Valery (2018) en Vensim para migrarlo hacia R, permitiendo la comparaci\'on de los resultados obtenidos en Vensim (Valery, 2018) contra el resultado obtenido del paquete en R. Buscando como meta la versatilidad a los investigadores interesados de poder trabajar sobre un ambiente que puedan manejar y modificar con las diferentes necesidades de su trabajo.\\




\section{Objetivos}

\subsection{Objetivo General}
Desarrollar un paquete en el lenguaje R implementando el modelo MOMOS.


\subsection{Objetivos Espec\'ificos}

\begin{itemize}
\item Estudiar la Estructura del modelo MOMOS.
\item Crear las funciones primarias en el lenguaje R de la implementaci\'on del modelo MOMOS.
\item Realizar las pruebas unitarias y funcionales del paquete creado en el lenguaje R.
\end{itemize}

\section{Justificaci\'on e Importancia}

En las ciencias agron\'omicas y \'areas afines, en la \'ultima d\'ecada, se han utilizado una serie de herramientas que permiten mejorar los procesos de producci\'on, así como del manejo de los recursos, entre las mismas, se encuentran los modelos din\'amicos de simulaci\'on los cuales logr\'an mostrar una representaci\'on lo m\'as cercana a la realidad, generando un mejor entendimiento de los procesos en los ecosistemas y agroecosistemas. \\

\section{Alcance y Limitaciones}

El paquete se desarrolla en el lenguaje R y permite simular el modelo MOMOS, el resultado obtenido se compar\'a con los obtenidos por Valery (2018) en el programa VENSIM para simular el mismo modelo. \\
