\chapter{Conclusiones y Recomendaciones}

\section{Conclusiones}

Se logr\'o desarrollar el paquete en lenguaje R sobre una herramienta de \textit{Software libre} para simular el modelo MOMOS con la finalidad de sustituir el uso del programa privativo VENSIM.\\

El estudio de la estructura del modelo MOMOS permiti\'o reescribir las funciones matem\'aticas como derivadas, ecuaciones diferencias, suma de cuadrados, entre otros en dicho lenguaje.\\

Tambi\'en fue posible crear pruebas unitarias que cubrieran totalmente el desarrollo planteado, usando diferentes librerias que facilitar\'an la realizaci\'on de dichas pruebas, asi mismo se llevo a cabo la ejecuci\'on de las pruebas funcionales mediante la corrida del paquete con datos simulados y datos reales, dando como resultado unos valores que pod\'ian ser ajustados. Para mejorar la diferencia entre los datos se procedi\'o a calibrar el modelo de un par\'ametro espec\'ifico \textit{Kresp}.\\

\section{Recomendaciones}

Se recomienda en una futura investigaci\'on calibrar otras par\'ametros importantes para el modelo.\\

Adicionalmente, es importante lograr capturar m\'as datos experimentales del modelo para tener un muestreo mayor para mejorar la presi\'on de la simulaci\'on.